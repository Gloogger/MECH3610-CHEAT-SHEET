\section{Week 10}
{\Large \textbf{{\color{red}\underline{First-Law Analysis of Combustion}}}}



Thermodynamic quantities are referenced to a standard state, usually at 298 K and 1 atm.

\begin{itemize}
    \item Use ideal gas law for combustions!
    \item First Law of Thermodynamics
    \begin{itemize}
        \item constant volume system
        \begin{align*}
            dU &= dQ - dW
        \end{align*}
        where $dU$ = change of internal energy, $dQ$ = heat input to the system, $dW$ = work done by the system on its surroundings.
        \item constant pressure system
        \begin{align*}
            dH &= dQ - dU
        \end{align*}
        where $dH$ = change of internal energy, $dQ$ = heat input to the system, $dW$ = work done by the system on its surroundings, except expansion work $pdV$.
    \end{itemize}
    
    \item 1st Law for steady flow system
    \begin{equation*}
        \dot{H}_o - \dot{H}_i = \dot{Q} - \dot{W}
    \end{equation*}
\end{itemize}

\begin{itemize}
    \item Partial pressure
    \begin{equation*}
        P_i = y_i \cdot P_m
    \end{equation*}
    \item Partial volume
    \begin{equation*}
        v_i = y_i \cdot V_m
    \end{equation*}
    \item Partial heat capacity
    \begin{equation*}
        c_{p,m} = \sum y_i \cdot c_{p,i}
    \end{equation*}
    \item Useful molecular weights:
    \begin{itemize}
        \item $W_H = 1$ kg/kmol, $W_{H_2} = 2$ kg/kmol;
        \item $W_C = 12$ kg/kmol;
        \item $W_O = 16$ kg/kmol;
        \item $W_N = 14$ kg/kmol.
        \item $W_{air} = W_{O_2} + 3.76\cdot W_{N_2} = 137.28$ kg/kmol
    \end{itemize}
    \item {\color{blue}\textbf{\ul{Calorific equation of state}}}
    \begin{itemize}
        \item are expressions relating energy variables to pressure and temperature:
        \begin{align*}
            u &= u (T,v) \; \text{internal energy} \\
            h &= h (T,P) = u + Pv \; \text{enthalpy}
        \end{align*}
        For ideal gases, pressure and volume dependence are ignored:
        \begin{align*}
            u &= u(T) \\
            h &= h(T)
        \end{align*}
    \end{itemize}
    \item {\color{blue}\textbf{\ul{Total enthalpy / Species enthalpy}}}
    \begin{itemize}
        \item Symbol: $h_{{\color{blue}t},{\color{red}m}}$ for {\color{blue}\textbf{t}}otal enthalpy of {\color{red}\textbf{m}}ixture; $\tilde{h}_{{\color{blue}t},{\color{red}m}}$ for mole based unit.
        \item Species / total enthalpy is used to quantify the energy change during chemical reaction and phase changes.
        \item By analogy to relative velocities,
        \begin{align*}
            h_{t,m} &= h_{f,m}^{0} + \Delta h_m \\
            v_B &= v_A + v_{B/A} \\
            \text{where } v_A &= \cancelto{0}{v_O} + v_{A/O}
        \end{align*}
    \end{itemize}
    \item {\color{blue}\textbf{\ul{Enthalpy of formation}}} 
    \begin{itemize}
        \item Symbol: $h_{{\color{blue}f},{\color{red}m}}^{{\color{brown}0}}$ for enthalpy of {\color{blue}\textbf{f}}ormation of a {\color{red}\textbf{m}}ixture measured at standard state of ${\color{brown}[P^{0},\, T^{0}] = [1\, \text{atm},\, 298 K]}$; $\tilde{h}_{{\color{blue}f},{\color{red}m}}^{{\color{brown}0}}$ for mole based unit.
        \item Enthalpy of formation is associated with energy in {\color{red}chemical bonds} or {\color{red}intramolecular bonds};
        \item Found by using Tables
    \end{itemize}
    \item {\color{blue}\textbf{\ul{Sensible enthalpy change}}}
    \begin{itemize}
        \item Symbol: $\Delta h_{{\color{red}m}}$ for sensible enthalpy change of a {\color{red}m}ixture.
        \begin{equation*}
            \Delta h_{m} = c_{p,m} \cdot (T - T^0)
        \end{equation*}
    \end{itemize}
    
 
    \item {\color{blue}\textbf{\ul{Types of Combustion}}}
    \begin{enumerate}
        \item {\color{brown}\textbf{Stoichiometric mixture:}} just the amount needed of the oxidiser to completely burn a quantity of fuel;
        \item {\color{brown}\textbf{Lean mixture:}} not enough fuel is supplied;
        \item {\color{brown}\textbf{Rich mixture:}} too much fuel is supplied (not enough oxidiser).
    \end{enumerate}
    \item Air-fuel ratio \& Equivalence ratio
    \begin{itemize}
        \item Air-fuel ratio (AFR):
        \begin{align*}
            \text{mass-based} \; \text{AFR} &= \frac{m_{air}}{m_{fuel}} \\
            \text{mole-based} \; \text{AFR} &= \frac{N_{air}}{N_{fuel}}
        \end{align*}
        \item Fuel-air ratio (FAR):
        \begin{align*}
            \text{mass-based} \; \text{FAR} &= \frac{m_{fuel}}{m_{air}} \\
            \text{mole-based} \; \text{FAR} &= N_{fuel} / N_{air}
        \end{align*}
        \item Equivalence ratio ($\phi$)
        \begin{align*}
            \phi &= \frac{(\text{AFR})_{st}}{\text{AFR}} = \frac{\text{FAR}}{(\text{FAR})_{st}}
        \end{align*}
        $\phi >1 \rightarrow$ rich, $\phi < 1 \rightarrow$ lean
        \item Relative air-to-fuel ratio ($\lambda$)
        \begin{equation*}
            \lambda = \frac{1}{\phi}
        \end{equation*}
        $\lambda > 1 \rightarrow$ lean, $\lambda < 1 \rightarrow$ rich.
        \item $\phi$ and $\lambda$ are the same regardless of mole-based or mass-based unit system.
    \end{itemize}
    \item Writing the non-stoichiometric equations
    \begin{itemize}
        \item Standard form:
        \begin{align*}
            &\text{CH}_4 + a (\text{O}_2 + 3.76 \text{N}_2 ) \\
            &\rightarrow b \text{CO}_2 + c \text{H}_2 O + d \text{N}_2
        \end{align*}
        \item want \textbf{\ul{per unit mole of fuel burned:}}
        \begin{align*}
            &\text{CH}_4 + \frac{a}{\phi} (\text{O}_2 + 3.76 \text{N}_2 ) \\
            &\rightarrow b \text{CO}_2 + c \text{H}_2 O + d \text{N}_2 \\
            &\text{CH}_4 + a \lambda (\text{O}_2 + 3.76 \text{N}_2 ) \\
            &\rightarrow b \text{CO}_2 + c \text{H}_2 O + d \text{N}_2
        \end{align*}
        \item want \textbf{\ul{per unit mole of oxygen burned:}}
        \begin{align*}
            &\phi \text{CH}_4 + a \lambda (\text{O}_2 + 3.76 \text{N}_2 ) \\
            &\rightarrow b \text{CO}_2 + c \text{H}_2 O + d \text{N}_2
        \end{align*}
    \end{itemize}
    \item Enthalpy of reaction: steady-flow energy equation (SFEE)
    \begin{align*}
        \Delta h_R &= q_{cv} = h_{prod} - h_{reac}
    \end{align*}
\end{itemize}

{\Large \textbf{{\color{red}\underline{Second-Law Analysis of Combustion}}}}

\begin{itemize}
    \item Why needed?
    \begin{itemize}
        \item First-law analysis requires the assumption of \textbf{\ul{complete combustion}}.
        \item In high-temperature combustion processes, the reaction can go inverse due to dissociation. High energy molecules are constantly breaking up and reforming.
        \item Second-law analysis is needed to figure out how much reverse reaction is going on.
    \end{itemize}
    \item {\color{red}Entropy} is a measure of {\color{blue}unavailability of a system's energy to do work}, and of the {\color{brown}quality of energy (high entropy = poor quality)}.
    \item Units
    \begin{align*}
        S \; \qquad &\text{(kJ/K)} \\
        s \; \qquad &\text{(kJ/kg$\cdot$K)} \\
        \tilde{s} \; \qquad  &\text{kJ/kmol$\cdot$K}
    \end{align*}
    \item \textbf{\ul{Gibbs energy}} is related to the {\color{blue}availability of energy} = maximal amount of non-expansion work that can be down by a thermodynamically closed system with no mass transfer across boundaries.
    \item Units
    \begin{align*}
        G &= H_t - TS \; \qquad \text{(kJ/K)} \\
        g_m &= h_{t,m} - T s_{t,m} \; \qquad \text{(kJ/kg)} \\
        \tilde{g}_m &= \tilde{h}_{t,m} - T \tilde{s}_{t,m} \; \qquad  \text{kJ/kmol}
    \end{align*}
\end{itemize}

\begin{itemize}
    \item {\color{blue}\textbf{\ul{Species/total Enthalpy}}}
    \begin{itemize}
        \item Definition
        \begin{align*}
            &\overline{s}_i (T, P_i) - \overline{s}^{0}_{i} (T^0 ,P^0 ) \\
            &= \int_{T^0}^{T} \frac{\overline{c}_{p,i}(T')}{T'} dT' - R_u \ln(\frac{P_i}{P_0})
        \end{align*}
        \item Useful form
        \begin{align*}
            s_{t,m} &= s_{m}^{0} + \Delta s_m
        \end{align*}
    \end{itemize}
    \item Entropy at standard state (found by tables)
    \begin{equation*}
        s_{m}^{0} = \sum y_i \cdot s_{m,i}^{0}
    \end{equation*}
    \item Change in entropy as results of temperature and pressure change
    \begin{align*}
        \Delta s_m &= c_{p,m} \cdot \ln\left(\frac{T_2}{T_1}\right) - {\color{blue}R_m} \cdot \ln\left(\frac{P_2}{P_1}\right)\\
        &= c_{p,m} \cdot \ln\left(\frac{T}{T_0}\right) - {\color{red}R_u} \cdot  \left[\sum y_i \cdot \ln(y_i)\right]
    \end{align*}
\end{itemize}

{\Large \textbf{{\color{blue}\underline{Assumptions used in calculations}}}}

\begin{itemize}
    \item Gas behaves like ideal gases;
    \item 1D model;
    \item Complete combustion;
    \item Combustion happens instantaneously at constant pressure;
    \item Fuel is introduced instantaneously with the specified conditions;
    \item The process is adiabatic (approximated using the adiabatic flame temperature);
    \item Assume no dissociation. If temperature is sufficiently high such that dissociation is significant, \textbf{or} when combustion is rich, use {color{red}\ul{second law analysis}}. If dissociation occurs, products are not just $CO_2$ and $H_2 O$;
    \item Either constant pressure (useful for most problems) or constant volume (rare) is assumed.
\end{itemize}

